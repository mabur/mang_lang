\documentclass[a4paper,12pt]{article}
\usepackage{amsmath}
\usepackage{amsfonts}
\usepackage{listings}
\title{Manglang}
\author{Magnus Burenius}
\begin{document}

\maketitle

\section{Minimal Language for Data}

We start by defining a minimal language for storing data, similar to Json. It consists of these primitive values:
\begin{itemize}
\item Numbers like: \lstinline|1.0|
\item Strings like: \lstinline|''Magnus''|
\item Booleans true and false
\end{itemize}
Values can be packaged in these collections:
\begin{itemize}
\item Lists like: \lstinline|[1, 2, 3]|
\item Dictionaries like: \lstinline|{x = 1, y = 2}|
\end{itemize}
Collections can be nested. We can easily transform this format to Json by replacing = with : and adding quotes arounds dictionary keys.

\section{Extended Language for Data}

We add the posibility to refer back to symbols:
\begin{itemize}
\item Reference to symbol in current scope like: \lstinline|{x = 1, y = x}|
\item Reference to symbol in parent scope like:  \lstinline|{x = 1, y = {z = x}}|
\item Reference to symbol in child scope like:  \lstinline|{x = {y = 1}, z = y}|
\end{itemize}
We add the posibility for conditionals:
\begin{itemize}
\item Conditionals like: \lstinline|if x then y else z|
\end{itemize}

\subsection{Evaluation}

The evaluation of a program takes an AST from the extended language for data and collapses it to an AST for the minimal laguage by:
\begin{enumerate}
\item Collapsing the references to the expressions they refer to.
\item Collapsing conditionals.
\end{enumerate}

\section{Minimal Language for Algorithms}

We will now extend our language to handle functions. We extend the abstract syntax tree with these nodes:
\begin{itemize}
\item Functions like: \lstinline|from x to [x, x]|
\item Reference to function input like: \lstinline|from x to [x, x]|
\item Function application like: \lstinline|{f = from x to [x, x], y = f 1}|
\end{itemize}

\subsection{Evaluation}

We evaluate the program by transforming the AST to another graph. We keep a copy of the original AST. Functions know about their environment in the graph.
\begin{enumerate}
\item We do not collapse: numbers, strings, booleans
\item We collapse conditionals.
\item We collapse references.
\item We collapse functions to bound functions. This is a function bound to an environment. Everything in the function body is untouched.
\item We collapse function calls. When collapsing a function call we evaluate the body of the function, given the bound environment and the input to the function.
\end{enumerate}
How to handle environments while evaluating? Do we want to link both up and down in the graph we get when evaluating the AST? Could be useful when collapsing references and function calls. We don't allow functions to survive their bound environment, i.e. to be the output.

http://craftinginterpreters.com/contents.html

\end{document}
