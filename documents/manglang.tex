\documentclass[a4paper,12pt]{article}
\usepackage{amsmath}
\usepackage{amsfonts}
\usepackage{listings}
\title{Manglang}
\author{Magnus Burenius}
\begin{document}

\maketitle

\section{Overiew}

Mang Lang is a functional language without any side efects. The lifetime of a Mang Lang program has four stages:
\begin{enumerate}
\item Program written in Mang Lang.
\item Program parsed into a tree graph.
\item Program evaluated into another tree graph.
\item Program result presented in Mang Lang.
\end{enumerate}

\section{Minimal Language for Data}

We start by defining a minimal language for storing data, similar to Json. It consists of these primitive values:
\begin{itemize}
\item Numbers like: \lstinline|1.0|
\item Strings like: \lstinline|''Magnus''|
\item Booleans true and false
\end{itemize}
Values can be packaged in these collections:
\begin{itemize}
\item Lists like: \lstinline|[1, 2, 3]|
\item Dictionaries like: \lstinline|{x = 1, y = 2}|
\end{itemize}
Collections can be nested. We can translate this format to Json by replacing = with : and adding quotes arounds dictionary keys.

\section{Extended Language for Data}

We add the posibility to refer back to symbols:
\begin{itemize}
\item Reference to symbol in current scope like: \lstinline|{x = 1, y = x}|
\item Reference to symbol in parent scope like:  \lstinline|{x = 1, y = {z = x}}|
\item Reference to symbol in child scope like:  \lstinline|{x = {y = 1}, z = y<x}|
\end{itemize}
We add the posibility for conditionals:
\begin{itemize}
\item Conditionals like: \lstinline|if x then y else z|
\end{itemize}

\subsection{Evaluation}

To evaluate a program written in the extended language for data we follow these steps:
\begin{enumerate}
\item Parse input written in the extended language for data. This results in a tree graph.
\item Collapse references and conditionals to the expressions they refer to. This results in another tree graph.
\item Serialize the evaluated tree to get the result written in the minimal language for data.
\end{enumerate}
We parse the input into a tree. In this tree a parent keeps references to its children, and each child also keeps a reference to its parent. We use this when collapsing references and conditionals.

\section{Minimal Language for Algorithms}

We will now extend our language to handle functions. We extend the abstract syntax tree with these nodes:
\begin{itemize}
\item Functions like: \lstinline|from x to [x, x]|
\item Reference to function input like: \lstinline|from x to [x, x]|
\item Function application like: \lstinline|{f = from x to [x, x], y = f 1}|
\end{itemize}

\subsection{Evaluation}

We evaluate the program by transforming the AST to another graph. We keep a copy of the original AST. Functions know about their environment in the graph.
\begin{enumerate}
\item We do not collapse: numbers, strings, booleans
\item We collapse conditionals.
\item We collapse references.
\item We do NOT collapse functions and do NOT collapse anything in the function body.
\item We collapse function calls. When collapsing a function call we evaluate the body of the function it refers to, given the input to the function.
\end{enumerate}
We don't allow functions to survive their bound environment, i.e. to be the output. We do not have closures.

\section{Implementation}

\subsection{Abstract Syntax Tree}
The parsing takes a string and builds an Abstract Syntax Tree which consist of the following nodes:
\begin{verbatim}
type Number = {value: float}
type String = {value: str}
type Boolean = {value: str}
type List = {values: list<Node>}
type Dictionary = {values: list<tuple<str, Node>>}
type Conditional = {condition: Node; a: Node; b: Node}
type LookupSymbol = {name: str}
type LookupChild = {name: str; child: Node}
type LookupFunction = {name: str; input: Node}
type Function = {input_name: str; body: Node}
\end{verbatim}

\subsection{Evaluation Tree}

The evaluation takes an Abstact Syntax Tree and builds an Evaluation Tree which consists of similar nodes but with a parent pointing to another Evaluation Node. The nodes in the Abstract Syntax Tree are cloned during the evaluation and given parents.

\subsection{Methods}

Each Node provides an evaluation method that generates new nodes:
\begin{verbatim}
evaluate(self: Node, parent: Node) -> Node
\end{verbatim}
Each Node provides a lookup method that search for a reference in the node and if nothing is find passes the query to its parent. The lookup does not generate new nodes:
\begin{verbatim}
lookup(self: Node, reference: str) -> Node
\end{verbatim}

\end{document}
